\documentclass{beamer}
\usepackage[spanish]{babel}
\usepackage[utf8]{inputenc}
\usepackage[T1]{fontenc}
\usepackage{lmodern}
\usepackage{graphicx}
\usepackage{ifthen}
\graphicspath{{figuras/}{./figuras/}{./}}

\title{Capítulo 1 – Introducción al Control}
\author{Jordan Blancas}
\date{\today}

\begin{document}

\begin{frame}
  \titlepage
\end{frame}

\begin{frame}{Conceptos básicos}
  \begin{itemize}
    \item Qué es un sistema de control
    \item Ejemplos en la vida real
    \item Importancia en ingeniería
  \end{itemize}
\end{frame}

% ⬇️ La imagen SIEMPRE dentro de un frame
\begin{frame}{Diagrama del capítulo}
  \centering
  \IfFileExists{figuras/cap1_diagrama.png}{
    \includegraphics[width=\linewidth]{figuras/cap1_diagrama.png}
  }{
    \fbox{\parbox{0.9\linewidth}{\centering Imagen no disponible: figuras/cap1_diagrama.png}}
  }
\end{frame}

\begin{frame}{Conclusiones}
  \begin{itemize}
    \item El control regula sistemas dinámicos
    \item Se basa en retroalimentación
    \item Base para temas más avanzados
  \end{itemize}
\end{frame}

\end{document}
